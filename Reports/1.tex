% !TeX spellcheck = en_GB
\documentclass[]{report}


% Title Page
\title{Simulatenous Localisation And Mapping for Unmanned Ground and Aerial Vehicles}
\author{P Sai Ramana Kiran}


\begin{document}
\maketitle

\begin{abstract}
	The purpose of this document is to familiarize the reader on the Simultaneous Localization and Mapping for robotics, particularly for Unmanned Ground Vehicles. This document starts with basic concepts of probability and goes through the research papers and discusses the concept along the way. 
	The main objective of this document is also to research about the SLAM which is compatible with planning and navigation. 
\end{abstract}

\newpage

\begin{center}{\section*{Motivation}}\end{center}

Simultaneous Localization And Mapping aka SLAM is basic problem in robotics. This addresses the problem of building a map of an unknown area from a sequence of landmark measurements obtained from moving a robot. Since there is an uncertainty in robot position as well as measurements it obtained while moving, the problem becomes cyclic in terms of mapping induces localization and, localization induces mapping and hence the name SLAM. 

\newpage
 \begin{center}
 	\section*{Introduction}
 \end{center}
\newpage
\tableofcontents
\newpage
\begin{center}
	\section*{Probabilistic Graphical Models}
\end{center}
 Graph theory and more specifically PGMs forms the basis for both computational and analytical formulation of SLAM
\newpage
 

\end{document}          
